\documentclass{beamer}
%\usetheme{CambridgeUS}
\usetheme{Szeged}
%\usetheme{Warsaw}
\usecolortheme{beaver}

\usepackage[utf8]{inputenc}

\usepackage[normalem]{ulem}

\title{What you learned in Calc 1}
\date{2019 Fall}

\begin{document}
\frame{\titlepage}

%\begin{frame}{Table of Contents}
  %\tableofcontents
%\end{frame}

\section{Overall}
\begin{frame}[label={sec:orgb61fcc0}]{Main Ideas of Calculus}
\begin{block}{The Derivative}
\begin{itemize}
\item "Instantaneous rate of change"
\item "Slope of a tangent line"
\item Limit of the secant line, as the end-points get \uline{arbitrarily close}
\end{itemize}
\end{block}
\begin{block}{The Integral}
\begin{itemize}
\item "Area under a curve"
\begin{itemize}
\item Approximate by summing up rectangles
\item Approximate better by using \uline{infinitely many} rectangles
\end{itemize}
\end{itemize}
\end{block}
\begin{block}{FTC (Fundamental Theorem of Calculus)}
Slogan: "Differentiation and Integration are inverse operations"
\end{block}
\end{frame}

\begin{frame}[label={sec:org833f693}]{Main Ideas of Calculus (Mathematically)}

\begin{block}{The Derivative}
\(f'(a) := \lim\limits_{h\to0} \frac{f(a+h) - f(a)}{h}\)
or 
\(f'(a) := \lim\limits_{x\to a} \frac{f(x) - f(a)}{x-a}\)
\end{block}

\begin{block}{The Integral}
\(\int_a^b f(x) dx := \lim\limits_{N\to\infty} \sum\limits_{i=1}^N f(x_i) \cdot \frac{b-a}{N}\)
\end{block}

\begin{block}{FTC (Fundamental Theorem of Calculus)}
\begin{itemize}
\item Part 1: \(\frac{d}{dx} \int_a^x f(t) dt = f(x)\)
\item Part 2: \(\int_a^b f(x) dx = F(b) - F(a)\)
\end{itemize}
\end{block}
\end{frame}

\begin{frame}[label={sec:org4009a94}]{Limits}
We want to be to evaluate limits. They are just a calculation.

\begin{itemize}
\item Start with direct substition
\item If that fails, try:
\begin{itemize}
\item Algebraic technique
\item L'Hopital's: 
\[
      \lim_{x\to a} \frac{f(x)}{g(x)} = \lim_{x\to a} \frac{f'(x)}{g'(x)}
      \]

\begin{itemize}
\item \alert{The LHS must be of type \(\frac00\) or type \(\frac{\infty}{\infty}\) for this to hold}
\end{itemize}
\end{itemize}
\end{itemize}

\begin{block}{Indeterminate forms}
\begin{itemize}
\item \(\frac00\), \(\frac{\infty}{\infty}\), \(\infty-\infty\), \(0\cdot\infty\)
\item \(1^\infty\), \(0^0\), \(\infty^0\)
\begin{itemize}
\item Note that taking \(\ln()\) of these reduces them to the type \(0 \cdot \infty\)
\end{itemize}
\end{itemize}
\end{block}
\end{frame}

\section{Differential Calculus}

\begin{frame}[label={sec:org5f23ff0}]{Derivative Rules}
\begin{block}{Basic Derivative rules}
\begin{itemize}
\item Product rule: \((fg)' = f'g + fg'\)
\item Quotient rule: \((\frac{f}{g})^{'} = \frac{f'g - fg'}{g^2}\)
\item Chain rule: \([f(g(x))]' = f'(g(x)) \cdot g'(x)\)
\item Power rule: $(x^n)' = n x^{n-1}$
  \begin{itemize}
  \item \alert{Do not confuse with: $\int x^n dx = \frac{x^{n+1}}{n+1} + C$}
  \end{itemize}
\end{itemize}
\end{block}
\begin{block}{Trig Derivatives}
\begin{itemize}
\item \(\sin \to \cos\), \(\cos \to -\sin\)
\item \(\tan \to \sec^2\), \(\cot \to -\csc^2\)
\item \(\sec(x) \to \sec(x) \tan(x)\), \(\csc(x) \to -\csc(x) \cot(x)\)
\end{itemize}
\end{block}
\end{frame}

\begin{frame}[label={sec:orga8d0a69}]{More Derivative Rules}
\begin{block}{Log and Exponential Derivatives}
\begin{itemize}
\item Special
\begin{itemize}
\item \(\ln x \to \frac 1x\)
\item \(e^x \to e^x\)
\end{itemize}
\item General
\begin{itemize}
\item \(\log_b x \to \frac{1}{\ln b} \frac{1}{x}\)
\item \(b^x \to (\ln b) b^x\)
\end{itemize}
\end{itemize}
\end{block}


\begin{block}{Inverse Trig Derivatives}
\begin{itemize}
\item \(\sin^{-1}x \to \frac{1}{\sqrt{1-x^2}}\), \(\cos^{-1} x \to -\frac{1}{\sqrt{1-x^2}}\)
\item \(\tan^{-1}x \to \frac{1}{x^2 + 1}\), \(\cot^{-1}x \to -\frac{1}{x^2+1}\)
\item \(\sec^{-1}x \to \frac{1}{|x| \sqrt{x^2-1}}\), \(\csc^{-1}x \to -\frac{1}{|x|\sqrt{x^2-1}}\)
\end{itemize}
\end{block}
\end{frame}

\begin{frame}[label={sec:org8e01da0}]{Logarithmic differentiation}
  \begin{block}{What is it}
    Taking ln() of both sides before differentiating (and remembering to undo this).
  \end{block}

  \begin{block}{Why use it}
    \begin{itemize}
    \item ln() will turn multiplication/division into addition/subtraction, so that you can avoid product rule and quotient rule.
    \item If \(x\) occurs in both the base and the power, then this technique is necessary (e.g. $y = x^x$).
    \end{itemize}
  \end{block}
\end{frame}

\begin{frame}[label={sec:orgfecd3ee}]{Implicit differentiation}
\begin{itemize}
\item Using \(y \to y'\) in addition to ordinary derivative rules.
\item Why?
\begin{itemize}
\item Sometimes both \(x\) and \(y\) are on the same side of the equation, and the equation cannot be solved for \(y\) fully.
\item However, we can still think about slopes of tangent lines for such curves, so we want to be able to compute the derivative
\item E.g. A circle. \(x^2 + y^2 = 1\)
\end{itemize}
\end{itemize}

Related rates are an example of this
\end{frame}

\begin{frame}[label={sec:org54f2f5a}]{Linearization}
\begin{itemize}
\item The idea behind it: Use the derivative as a linear approximation for your function.

\item To construct an \(L(x)\) that approximates \(f(x)\) near \(x=a\):
\[
    L(x) = f(a) + f'(a)(x-a)
    \]
\end{itemize}
\end{frame}

\begin{frame}[label={sec:orgc1829e8}]{Curve sketching}
\begin{itemize}
\item \(f'\) denotes increasing/decreasing
\begin{itemize}
\item positive: increasing
\item negative: decreasing
\item zero or undefined: critical point
\end{itemize}

\item \(f''\) denotes concavity
\begin{itemize}
\item positive: concave up
\item negative: concave down
\item zero: inflection point
\end{itemize}
\end{itemize}
\end{frame}

\begin{frame}[label={sec:orgbd73406},fragile]{Optimization}
  Overall: Want to find critical points, and classify as local min/max.

  To classify:
\begin{enumerate}
\item First derivative test:
Evaluate \(f(x)\) \alert{on either side} of the critical point
\begin{itemize}
\item \texttt{+|-} : max
\item \texttt{-|+} : min
\item otherwise : neither
\end{itemize}

\item Second derivative test:
Evaluate \(f''(p)\) \alert{at} the critical point

\begin{itemize}
\item \(f''(p)>0\) : Concave up, so local min
\item \(f''(p)<0\) : Concave down, so local max
\item \(f''(p)=0\) : Inconclusive
\end{itemize}
\end{enumerate}

Making the case that a local min/max is a \alert{global} min/max requires an extra condition:
\begin{enumerate}
\item First derivative test: \(f(x)\) has no other critical points
\item Second derivative test: \(f'' > 0\) always or \(f'' < 0\) always.
\end{enumerate}
\end{frame}

\section{Integral Calculus}

\begin{frame}[label={sec:orgddabff4}]{Integral techniques}
\begin{itemize}
\item Spotting the antiderivative by inspection
\item U-sub
\end{itemize}
\end{frame}

\begin{frame}{Area between curves}
  ``Upper function $-$ lower function''
\end{frame}

\begin{frame}[label={sec:org7499686}]{Volume of revolution}
\begin{block}{Disk / Washer method}
\begin{itemize}
\item Disk: 
\[
    V = \pi \int_a^b [r(x)]^2\ dx
    \]
\item Washer: 
\[
     V = \pi \int_a^b \left( [R(x)]^2 - [r(x)]^2 \right) \ dx
     \]
\end{itemize}
\end{block}
\begin{block}{Shell method}
\[
   V = 2\pi \int_a^b r(x) \cdot h(x) \ dx
   \]
\end{block}
\end{frame}

\end{document}
