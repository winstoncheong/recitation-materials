\documentclass[10pt]{scrartcl}

\usepackage[inline]{enumitem}
\usepackage{graphicx}
\usepackage{float}
\usepackage{booktabs}
\usepackage{tcolorbox}
\usepackage{textcomp}
\usepackage{commath}
\usepackage{mathtools}
\usepackage{physics}
\usepackage{outlines}
\usepackage[colorlinks=true]{hyperref}


\usepackage{siunitx}
\DeclareSIUnit{\feet}{ft}

\newcommand{\ZZ}{\mathbb{Z}}
\newcommand{\vavg}{v_{\text{avg}}}
\newcommand{\inv}{^{-1}}

\everymath{\displaystyle} 

\title{Exam 2 Prep}
\date{Last Modified: 2019/10/18}
\author{}

\begin{document}
\maketitle
\tableofcontents

%\fbox{The shaded boxed are full solutions, as the answer \emph{is} the whole solution.}
%\vskip 1em

\section{Material outline:}
\begin{outline}
  \1 Previous material:
  \2 Unit circle
  \2 Product rule
  \2 Quotient rule
  \2 Chain rule
  \2 Trig derivatives
  \1 Derivatives of log and exponential:
  \begin{align*}
    \dod{}{x} \ln x &= \frac 1x & \dod{}{x} e^x &= e^x\\
    \dod{}{x} \log_b x &= \frac{1}{\ln b} \frac 1x & \dod{}{x} a^x &= (\ln a) a^x
  \end{align*}
  \1 Derivatives of inverse trig functions:
  \begin{align*}
    \dod{}{x} \sin\inv x &= \frac{1}{\sqrt{1 - x^2}} & \dod{}{x} \cos\inv x &= -\frac{1}{\sqrt{1 - x^2}} \\
    \dod{}{x} \tan\inv x &= \frac{1}{x^2 + 1} & \dod{}{x} \cot\inv x &= -\frac{1}{x^2 + 1}\\
    \dod{}{x} \sec\inv x &= \frac{1}{|x|\sqrt{x^2 - 1}} & \dod{}{x} \csc\inv x &= -\frac{1}{|x|\sqrt{x^2 - 1}}
  \end{align*}
  \1 Logarithmic differentiation
  \1 Implicit differentiation
  \1 Related rates
  \1 Linearization
  \1 Differentials
  \1 Mean value theorem?
  \1 Finding / identifying local and global extrema on a graph
  \1 The process of finding absolute min/max on a closed domain
  \1 First and second derivative tests
  \1 Calculating $\lim_{x\to\infty}$ and $\lim_{x\to-\infty}$
  \1 Curve sketching
  \1 Reading data off of a graph
  \2 Intervals where increasing/decreasing
  \2 Local mins and maxes
  \2 Intervals where concave up/down
\end{outline}

\section{Past exam problems}
The following problems all have written solutions in my exam archive at \url{http://www.math.ksu.edu/\~winstonc/exams/calc1}.

Note, I have rephrased some of the problems here, to ease typing. 
Unless I've made a mistake, the exam listed should be where the problem was originally found.

\subsection{Take Derivatives}
\begin{enumerate}
\item (2018 Fall Exam 2) $\dod{}{x}\left(x \cdot \arctan(3x^2)\right)$
\item (2018 Fall Exam 2) $\dod{}{x}\left(x \cdot \frac{2^x - \ln x}{e^x + 1} \right)$
\item (2018 Fall Final) $\dod{}{x}\left( \frac{\tan x}{\ln x + 3} \right)$
\item (2018 Fall Final) $\dod{}{x} \sin(x^2) \cdot \arctan(x)$
\item (2017 Fall Exam 2) $w(x) = \frac{\tan(e^x)}{1+x^3}$
\item (2017 Fall Exam 2) $h(x) = \sin\inv(\ln(x))$
\item (2017 Summer Exam 2) $f(x) = \tan(\ln(6x^4+x^2))$
\item (2017 Summer Exam 2) $y = e^{\sin x}$
\item (2017 Spring Exam 2) $\dod{}{x}\left( 2^x - \frac{5}{x^2} + \ln(7) \right)$
\item (2017 Spring Exam 2) $\dod{}{x}\left(\sqrt x \cdot \tan x\right)$
\item (2017 Spring Exam 2) $\dod{}{x} \arctan(2t^2 - 3)$
\item (2017 Spring Exam 2) $\dod{}{x} \left( \frac{6\ln x-2x^3}{e^x+3} \right)$
\item (2016 Spring Exam 2) $\dod{}{x} \sin x \cdot 2^x$
\item (2016 Spring Exam 2) $\dod{}{x}  \frac{\sqrt{x}}{\cos x}$
\item (2016 Spring Exam 2) $\dod{}{w} \arctan(5w^2 + 3)$
\item (2016 Spring Exam 2) $\dod{}{x}(e^{e^x})$
\item (2016 Spring Exam 2) $\dod{}{\theta} \tan\theta \cdot \ln\theta$
\item (2016 Fall Exam 2) $w(x) = x^3 e^{1/x}$
\item (2016 Fall Exam 2) $h(x) = \tan\inv(x^2)$
\end{enumerate}

\subsubsection{Using Logarithmetic Differentiation}
\textbf{Question}: If you weren't prompted, would you know that these problems involve logarithmic diff just by looking?

\begin{enumerate}
\item (2018 Fall Exam 2) Find the derivative of $h(x) = x^{5x^2}$
\item (2018 Fall Final) Find the derivative of $f(x) = x^{3x}$
\item (2017 Fall Exam 2) $y = x^{5x}(2 + 3x^2)^4$
\item (2017 Summer Exam 2) Find the derivative of $y = \frac{(x+2)^2}{(x+5)(3x-4)}$.
\item (2017 Spring Exam 2) Find the derivative of $h(x) = e^x \cdot x^{5\cos(x)}$
\item (2016 Spring Exam 2) Find the derivative of $h(x) = \frac{x^x}{(3x^2+4)^5}$
\item (2016 Spring Exam 2) Find $\dod yx$ if $y = x^{2x}(1-x)^7$.
\end{enumerate}

\subsubsection{Using Implicit Differentiation}
\begin{enumerate}
\item (2018 Fall Exam 2) Compute $\dod{y}{x}$ for $x^2y - e^y = x + 1$
\item (2018 Fall Final) Find $\dod{y}{x}$ for $x^3 + xy + y^4 = 5$
\item (2017 Fall Exam 2) Find an equation for the tangent line to the curve $x^2 y + y^2 = x^3 - 3$ at the point $(2, 1)$.
\item (2017 Summer Exam 2) Compute $\dod{y}{x}$ for $3xy^2 = y^3 - \cos x$
\item (2017 Spring Exam 2) Find $\dod{y}{x}$ if $\sin(xy^2) = x^2$.
\item (2016 Spring Exam 2) Find $\dod yx$ if $x^2y^3 = e^x - y^2$. 
\item (2016 Fall Exam 2) Find an equation for the tangent line to the curve $xy + 7 = x^3 + y^3$ at the point $(2, 1)$.
\item Find an equation of the tangent line to the curve $x^2 y^3 - x^3 y^2 = 4$ at the point $(1, 2)$. 

\end{enumerate}

\subsection{Linearize and approximate}
\begin{enumerate}
\item (2018 Fall Exam 2) Find the linearization of $g(x) = \sqrt{x}$ at $x = 25$. Use this to estimate $\sqrt{26}$.
\item (2017 Fall Exam 2) Find the linear approximation of $f(x) = \cos(x)$ near $x = \frac{\pi}{4}$. Use this to estimate $\cos(\frac\pi4+\frac{1}{10})$.
\item (2018 Fall Final) Use the linearization of $u(x) = \ln x$ at $x = 1$ to approximate $\ln(0.9)$.
\item (2017 Spring Exam 3) Find linearization for $g(x) = \ln(x)$ at $x = 1$. Use it to approximate $\ln(1.15)$.
\item (2016 Spring Exam 3) Find the linearization of $w(x) = \sqrt{x}$ at $x = 9$. Use it to estimate $\sqrt{9.6}$.
\item (2016 Fall Exam 2) Find the linear approximation of $f(x) = \sqrt[3]{x}$ near $x = 8$. Use it to estimate $\sqrt[3]{8.1}$.
\item (2015 Fall Exam 2) Find an equation of the tangent line to the curve $y = \sqrt[5]{x}$ at $x = 32$, and use this to approximate $\sqrt[5]{33}$.
\end{enumerate}

\subsection{Differentials}
\begin{enumerate}
\item (2018 Fall Exam 2) Let $V$ denote the volume of a cube of side length $x$. Find the differential $dV$ in terms of $x$ and $dx$.
\item (2017 Fall Exam 2) The volume of a cone of height 9 ft is given by $V = 3\pi r^2$. Estimate the change in volume using the differentials $dV$ and $dr$, if $r = 5$ ft and increased by $\frac{1}{10}$ ft.
\item (2017 Spring Exam 3) The volume of a sphere is given by $V = \frac 43 \pi r^3$. Find the differential $dV$.
\item (2016 Fall Exam 2) The volume of a sphere is given by $V = \frac 43 \pi r^3$. Estimate the change in volume by calculating $dV$, given that $r = 3$ inches and $dr = \frac{1}{12\pi}$ inches.
\item (2016 Spring Exam 3) Find $dy$ if $y = \cos(4x^2)$.
\end{enumerate}

\subsection{Find absolute max/min on a closed interval}
\begin{enumerate}
\item (2018 Fall Exam 2) $w(x) = x + \sin x$ on $[0, 2\pi]$.
\item (2018 Fall Final) $w(x) = x - \sqrt{x}$ on $[0, 4]$.
\item (2017 Fall Exam 2) Determine the absolute minimum and absolute maximum value of the function $f(x) = x^4 - 2x^2$ over the interval $[0, 2]$.
\item (2017 Spring Exam 3) Find the absolute min and max of $w(x) = 2x^3 - 9x^2 + 3$ on $[-1, 1]$.
\item (2016 Spring Exam 3) Find absolute min/max of $g(x) = x^3 - 3x^2 + 4$ on $[-1, 1]$.
\item (2016 Fall Exam 2) Find absolute min/max value of $f(x) = x^3 - 3x$ over the interval $[0, 2]$.
\item (2015 Fall Exam 2) Find the $x$ and $y$ coordinates of the absolute maximum of the function $y = x^3 - x^2 - x$ on the closed interval $0 \le x \le 2$ 
\end{enumerate}

\subsection{Horizontal Asymptotes}
\begin{enumerate}
\item (2018 Fall Exam 2) Find $\lim_{x\to-\infty} \frac{3x^9 - 7x + 3}{2 + 5x + 6x^9}$.
\item (2017 Fall Exam 2) Evaluate $\lim_{x\to-\infty} \frac{7-3x^5}{2x^5-15x^3}$.
\item (2016 Fall Exam 2) Evaluate $\lim_{x\to-\infty} \frac{x+3}{\sqrt{x^2-1}}$
  \item (2015 Spring Exam 1) Find horizontal asymptote(s) for $y = \frac{\sqrt{9x^2+5}}{2x-7}$
\end{enumerate}

\subsection{Related Rates}
\textbf{\color{red}{Always include units!}}
\begin{enumerate}
  \item (2018 Fall Exam 2) A hot air balloon rising vertically is tracked by an observer located 2 miles from the lift-off point. At a certain moment, the angle between the observer's line of sight and the horizontal is $\tfrac\pi4$, and it is changing at a rate of $\tfrac{1}{10}$ radians/minute. How fast is the balloon rising at this moment?
  \item (2018 Fall Final) A 5-foot ladder rests against a wall. The bottom of the ladder slides away from the wall at a rate of 2 ft/s. How fast is the top of the ladder sliding down the wall when the bottom of the ladder is 3 ft from the wall? 
  \item (2017 Fall Exam 2) A rocket is launched vertically upward from a point $P$. At the same time, a car is driving on a straight line away from the point $P$. Use related rates to determine the rate that the distance between the rocket and the car is increasing at the instant when the rocket is 3 miles up and travelling 500 miles per hour, and the car is 4 miles away from $P$ and travelling 30 miles per hour. (Assume the ground is flat.)
  \item (2017 Summer Exam 2) A plane is flying away from you at 500 mph at a height of 3 miles. How fast is the plane's distance from you increasing at the moment when the plane is flying over a point on the ground 4 miles from you?
  \item (2017 Spring Exam 2) Boyle's Law says that when a sample of gas is compressed at a constant temperature, the pressure $P$ and the volume $V$ satisfy the equation $PV = C$, where $C$ is a constant. Suppose that at a certain instant, the volume is $\SI{300}{cm^3}$, the pressure is 100 kPa, and the pressure is increasing at a rate of 20 kPa/min. At what rate is the volume changing at this instant?
  \item (2017 Spring Exam 2) The length of a rectangle is increasing at a rate of 2 ft/s, and its width is increasing at a rate of 3 ft/s. At what rate is the area of the rectangle increasing when the length is 6 ft and the width is 7 ft?
\end{enumerate}

\subsection{Curve sketching}
\begin{tcolorbox}
By ``the full process for curve sketching'' I mean:
\begin{enumerate}
\item Finding intercepts
\item Finding asymptotes (HA, VA)
\item Finding critical numbers and sign chart for $f'$
\item Finding inflection points and sign chart for $f''$
\item Sketching the curve using the above data.
\end{enumerate}
\end{tcolorbox}

\begin{enumerate}
\item (2018 Fall Exam 2) Sketch $f(x) = x^2(x+3)$, which has derivatives $f'(x) = 3x(x+2)$, $f''(x)=6(x+1)$. Do the entire process.
\item (2017 Fall Exam 2) For $g(x) = 20x^3 - 3x^5$, draw the number line for $g'(x)$ and $g''(x)$. Classify the critical points.
\item (2017 Fall Exam 2) Let $f(x) = \frac{2-x}{x^2 - 1}$. Then $f'(x) = \frac{x^2- 4x+1}{(x^2-1)^2}$. Find asymptotes, intercepts, critical points. Classify the critical points.
\item (2017 Summer Exam 2) Let $f(x) = x^3 - 2x^2 + x - 1$. Find when $f$ is concave up/down, increasing/decreasing, the inflection points, the critical points, and the local minima and maxima. 
\item (2017 Spring Exam 3) Sketch $f(x) = \frac{x^2}{x^2 + 3}$. It has $f'(x) = \frac{6x}{(x^2+3)^2}$ and $f''(x) = \frac{-18(x^2-1)}{(x^2+3)^3}$. Do the entire process.
\item (2016 Spring Exam 3) Sketch $f(x) = \frac{x^2- 1}{x^2 + 3}$. It has $f'(x) = \frac{8x}{(x^2+3)^2}$ and $f''(x) = \frac{-24(x^2-1)}{(x^2+3)^3}$. Do the entire process.
\item (2016 Fall Exam 2) Let $g(x) = x^5 - \frac{20}{3} x^3$. Draw the number line for $g'$ and $g''$. Classify the critical points. State when $g(x)$ is concave up.
\item (2016 Fall Exam 2) Let $f(x) = \frac{2x^2-6x}{4-x^2}$. Then $f'(x) = \frac{-2(3x^2-8x+12)}{(4-x^2)^2}$. Sketch $f(x)$ (skip using $f''$ or inflection points).
\item (2015 Fall Exam 3) Let $f(x) = \left( \frac{x-2}{x} \right)^2$. Then $f'(x) = \frac{4(x-2)}{x^3}$ and $f''(x) = \frac{-8(x-3)}{x^4}$. Do the entire process.
\end{enumerate}

\end{document}
